\usepackage{fontspec}
\usepackage{xunicode,xltxtra} %пакеты для работы xelatex
\setmainfont{Times New Roman} %выбор шрифта

\usepackage{setspace}

\usepackage{amsmath, amsfonts, amssymb} %математические пактеы
\usepackage{geometry} %разметка страницы
\geometry{
	lmargin=2.5cm, %левое поле
	rmargin=1cm, %правое поле
	tmargin=2cm, %верхнее поле
	bmargin=2cm, %нижнее поле
	headheight=14pt,
	headsep=1cm,
	footskip=1cm
}

\usepackage{indentfirst} 
\renewcommand{\normalsize}{\fontsize{14pt}{21pt}\selectfont} %установка кегля текста в 14

\usepackage{graphics} %пакет для вставки рисунков

\sloppy
\clubpenalty=10000
\widowpenalty=10000

\parindent=1.2cm %абзацный отступ

\usepackage{longtable} %пакет для работы с такблицами
\usepackage{makecell} %пакет для работы с такблицами
\usepackage{multirow} %пакет для работы с такблицами

\usepackage{caption} %пакте для работы с подписями рисунков и таблиц
\captionsetup[table]{name=Таблицa ,
	position=top, 
	belowskip=0pt,
	aboveskip=0pt,
	format = plain,
	justification=raggedright,
	labelsep=endash}
\captionsetup[figure]{name=Рисунок ,
	position=above,
	belowskip=0pt,
	aboveskip=0pt,
	format = plain,
	justification=raggedright,
	labelsep=endash}

\usepackage{graphicx} %пакет для работы с рисунками
\usepackage{tikz} %пакет для рисования в latex
\usepackage{xcolor} %пакет для работы с цветом
\usepackage{colortbl} %пакет для работы с цветом в таблицах
\usetikzlibrary{arrows} %библиотека для работы со стрелками в рисуемых схемах

\newtheorem{opr}{Определение}[section] 
\newtheorem{lem}{Лемма}[section]
\newtheorem{utv}{Утверждение}[section]
\newtheorem{zam}{Замечания}[section]
\newtheorem{teor}{Теорема}[section]

\usepackage{titlesec} %пакет для заголовков
\usepackage{tcolorbox} %пакет для работы с цветами

\usepackage{url}

\titleformat{\section}{\fontsize{16pt}{24pt}\selectfont\bfseries}{\thesection.}{.5em}{}
\titleformat{\subsection}{\normalsize\bfseries}{\thesubsection.}{.5em}{}
\titleformat{\subsubsection}{\normalsize\bfseries}{\thesubsubsection.}{.5em}{}

%определение описания нумерованных спсиков
\renewcommand{\labelenumi}{\arabic{enumi})}
\renewcommand{\labelenumii}{\arabic{enumi}.\arabic{enumii})}
\renewcommand{\labelenumiii}{\arabic{enumi}.\arabic{enumii}.\arabic{enumiii})}
\renewcommand{\labelenumiv}{\arabic{enumi}.\arabic{enumii}.\arabic{enumiii}.\arabic{enumiv})}

\usepackage{algorithm} %пакеты для работы с алгоритмами
\usepackage{algpseudocode}

\usepackage{xcolor}

\usepackage{listings}
\lstloadlanguages{C, Python, [x86masm]{Assembler}, bash, Java}
\lstset{language=[LaTex] Tex,
	extendedchars=true,
	%escapechar=|,
	%frame=tb,
	numbersep=5pt,
	commentstyle=\itshape,
	stringstyle=\bfseries,
	keepspaces=true,
	showstringspaces=false,
	tabsize=2,
	breaklines=true,
	basicstyle=\ttfamily\normalsize
}

\renewcommand{\refname}{Список литературы}
\renewcommand{\contentsname}{Cодержание}
\usepackage[linktoc=all]{hyperref}

\usepackage{sectsty}
\sectionfont{\LARGE\centering}
\subsectionfont{\Large\centering}
\subsubsectionfont{\centering}